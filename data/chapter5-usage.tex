% !Mode:: "TeX:UTF-8"
\chapter{论文不足}
\section{行文结构部分混乱}
在论述联合对抗训练的具体方法后,文章转讨论了“域调节”的数学理论,尝试用一些数学的理论对联合对抗训练提供理论支撑,但这套理论
在简单的描述后,就将具体的实验放入附录中,思维上存在跳跃性,导致读者无法对“域调节”这一数学理论有基本的认识,难以理解其与联合
对抗训练之间的关系,显得较为突兀。

\section{创新性不足}
本文以实验为主导,且带有一定的综述性质。对当前领域的学者具有较高的参考价值,但相对的,在创新性方面略显不足,论文的两个
主要创新点均为模型扰动攻击和训练时的小技巧,并非宏观上的模型级的创新。

\section{R+FGSM扰动攻击的局限性}
R+FGSM扰动攻击通过在扰动攻击中引入随机分量,使得扰动攻击有一定概率绕过梯度掩盖(由模型损失函数在数据点处高度非线性导致),
从而达到较好的攻击效果。但这仅限于黑盒攻击,在白盒攻击中传统的扰动攻击依然占优。

\section{联合对抗训练的局限性}
联合对抗训练虽能较好地提升模型对于黑盒攻击的鲁棒性,但由于在训练过程中需要对所有预训练模型进行扰动计算和对抗样本的生成,
会导致模型训练的计算成本迅速升高。另一方面,在联合对抗训练的效果验证环节,采用与训练模型高度相似的模型产生的扰动攻击,其验证
的过程与对抗训练的过程是否具有耦合关系尚待进一步探究。
