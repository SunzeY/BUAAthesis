% !Mode:: "TeX:UTF-8"
\chapter{解决方法}

如图\ref{fig:outline}所示,针对前一章所述的存在的问题,本文相对应地提出了三个解决方案。

\begin{figure}[htbp]
    \centering
    \includegraphics[width=0.6\textwidth]{figure/outline.pdf}
    \caption{问题及相对应的解决方法}
    \label{fig:outline}
\end{figure}

\section{可视化“极值退化现象”——正交梯度图}
在当前输入的梯度方向和梯度的垂直方向上可视化模型的损失函数,即以当前梯度方向和梯度的正交方向为轴,以扰动在两个方向上的投影长度为单位,绘制损失函数,用损失函数在数据点处出现的“褶皱”诠释“梯度掩盖”和“极值退化”现象产生的原因。具体公式如\eqref{a},其中 $g$为梯度方向的单位向量,$g^{\bot}$为与梯度方向相垂直的单位向量,$x$,$y$为在这两个方向上的投影长度,$s_0$为数据点。
\begin{equation}
L = L(s_0 + xg + yg^{\bot})
\label{a}
\end{equation}

利用这样的方法,可以将很高维度的张量输入投影到两个合理的方向,绘制二位曲面图观察损失函数界面,
从而观察不同模型的非线性程度以及在数据点出的褶皱情况。具体效果参看第四章实验部分。



\section{“R+FGSM”随机化扰动攻击}
针对完全基于梯度的扰动攻击可能存在的极值退化问题,提出一种新的攻击方法R+FGSM,即在梯度的方向上定长的攻击(FGSM)的基础上引入一定的随机分量(R),使得攻击有一定概率绕过“梯度掩盖”,达到更好的进攻效果。
FGSM扰动攻击的产生方式如\eqref{b},$x$为输入feature,$L$为损失函数,$y_{true}$为正确标签,$h$为网络模型。
\begin{equation}
x_{FGSM}^{adv} = x + \varepsilon \cdot sign(\nabla_x L(h(x), y_{true}))  
\label{b}
\end{equation}

在此基础上,本文提出的R+FGSM扰动攻击与FGSM保持同样的扰动步长$\varepsilon$,在其中引入一个随机的分量\eqref{eq_random_var}
\begin{equation}
x' = x + \alpha \cdot sign(N(0^d, I^d))
\label{eq_random_var}
\end{equation}
\begin{equation}
x^{adv} = x' + (\varepsilon - \alpha) \cdot sign(\nabla_{x'} L(h(x'), y_{true}))  
\label{R_FGSM}
\end{equation}


\section{联合对抗训练}
如图\ref{fig:ensemble_training2}所示,将人工智能比赛中常用的“联合训练”的思想引入对抗样本的生成。用预训练的小模型产生的扰动攻击增强当前训练的大模型的训练数据,以达到联合对抗训练的效果,实现模型和扰动的解耦。使得模型对于黑盒攻击更加健壮。具体而言,首先预先训练相同输入输出的小模型,由它们生成扰动攻击,再将扰动的数据覆盖到当前训练的大模型的输入样本上,产生增强的对抗样本,最后将对抗样本加入训练样本集,对大模型进行训练。
\begin{figure}[htbp]
    \centering
    \includegraphics[width=0.8\textwidth]{figure/ensemble_training2.pdf}
    \caption{联合对抗训练流程图}
    \label{fig:ensemble_training2}
\end{figure}


具体训练的流程如图\ref{fig:ensemble_training}所示,在对抗训练时加入一个新的超参数n,
以n个batch为循环单位。每个单位中,第一个batch使用基于原模型产生的扰动攻击生成对抗样本,
其余n-1个batch从预训练好的模型中随机选取,并基于该预训练模型产生扰动攻击生成对抗样本。
最后在验证环节,引入Holdout Model对训练效果进行验证。
\begin{figure}[htbp]
    \centering
    \includegraphics[width=0.8\textwidth]{figure/ensemble_training.pdf}
    \caption{联合对抗训练流程图}
    \label{fig:ensemble_training}
\end{figure}